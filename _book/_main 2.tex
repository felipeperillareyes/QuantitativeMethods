% Options for packages loaded elsewhere
\PassOptionsToPackage{unicode}{hyperref}
\PassOptionsToPackage{hyphens}{url}
%
\documentclass[
]{book}
\usepackage{amsmath,amssymb}
\usepackage{iftex}
\ifPDFTeX
  \usepackage[T1]{fontenc}
  \usepackage[utf8]{inputenc}
  \usepackage{textcomp} % provide euro and other symbols
\else % if luatex or xetex
  \usepackage{unicode-math} % this also loads fontspec
  \defaultfontfeatures{Scale=MatchLowercase}
  \defaultfontfeatures[\rmfamily]{Ligatures=TeX,Scale=1}
\fi
\usepackage{lmodern}
\ifPDFTeX\else
  % xetex/luatex font selection
\fi
% Use upquote if available, for straight quotes in verbatim environments
\IfFileExists{upquote.sty}{\usepackage{upquote}}{}
\IfFileExists{microtype.sty}{% use microtype if available
  \usepackage[]{microtype}
  \UseMicrotypeSet[protrusion]{basicmath} % disable protrusion for tt fonts
}{}
\makeatletter
\@ifundefined{KOMAClassName}{% if non-KOMA class
  \IfFileExists{parskip.sty}{%
    \usepackage{parskip}
  }{% else
    \setlength{\parindent}{0pt}
    \setlength{\parskip}{6pt plus 2pt minus 1pt}}
}{% if KOMA class
  \KOMAoptions{parskip=half}}
\makeatother
\usepackage{xcolor}
\usepackage{color}
\usepackage{fancyvrb}
\newcommand{\VerbBar}{|}
\newcommand{\VERB}{\Verb[commandchars=\\\{\}]}
\DefineVerbatimEnvironment{Highlighting}{Verbatim}{commandchars=\\\{\}}
% Add ',fontsize=\small' for more characters per line
\usepackage{framed}
\definecolor{shadecolor}{RGB}{248,248,248}
\newenvironment{Shaded}{\begin{snugshade}}{\end{snugshade}}
\newcommand{\AlertTok}[1]{\textcolor[rgb]{0.94,0.16,0.16}{#1}}
\newcommand{\AnnotationTok}[1]{\textcolor[rgb]{0.56,0.35,0.01}{\textbf{\textit{#1}}}}
\newcommand{\AttributeTok}[1]{\textcolor[rgb]{0.13,0.29,0.53}{#1}}
\newcommand{\BaseNTok}[1]{\textcolor[rgb]{0.00,0.00,0.81}{#1}}
\newcommand{\BuiltInTok}[1]{#1}
\newcommand{\CharTok}[1]{\textcolor[rgb]{0.31,0.60,0.02}{#1}}
\newcommand{\CommentTok}[1]{\textcolor[rgb]{0.56,0.35,0.01}{\textit{#1}}}
\newcommand{\CommentVarTok}[1]{\textcolor[rgb]{0.56,0.35,0.01}{\textbf{\textit{#1}}}}
\newcommand{\ConstantTok}[1]{\textcolor[rgb]{0.56,0.35,0.01}{#1}}
\newcommand{\ControlFlowTok}[1]{\textcolor[rgb]{0.13,0.29,0.53}{\textbf{#1}}}
\newcommand{\DataTypeTok}[1]{\textcolor[rgb]{0.13,0.29,0.53}{#1}}
\newcommand{\DecValTok}[1]{\textcolor[rgb]{0.00,0.00,0.81}{#1}}
\newcommand{\DocumentationTok}[1]{\textcolor[rgb]{0.56,0.35,0.01}{\textbf{\textit{#1}}}}
\newcommand{\ErrorTok}[1]{\textcolor[rgb]{0.64,0.00,0.00}{\textbf{#1}}}
\newcommand{\ExtensionTok}[1]{#1}
\newcommand{\FloatTok}[1]{\textcolor[rgb]{0.00,0.00,0.81}{#1}}
\newcommand{\FunctionTok}[1]{\textcolor[rgb]{0.13,0.29,0.53}{\textbf{#1}}}
\newcommand{\ImportTok}[1]{#1}
\newcommand{\InformationTok}[1]{\textcolor[rgb]{0.56,0.35,0.01}{\textbf{\textit{#1}}}}
\newcommand{\KeywordTok}[1]{\textcolor[rgb]{0.13,0.29,0.53}{\textbf{#1}}}
\newcommand{\NormalTok}[1]{#1}
\newcommand{\OperatorTok}[1]{\textcolor[rgb]{0.81,0.36,0.00}{\textbf{#1}}}
\newcommand{\OtherTok}[1]{\textcolor[rgb]{0.56,0.35,0.01}{#1}}
\newcommand{\PreprocessorTok}[1]{\textcolor[rgb]{0.56,0.35,0.01}{\textit{#1}}}
\newcommand{\RegionMarkerTok}[1]{#1}
\newcommand{\SpecialCharTok}[1]{\textcolor[rgb]{0.81,0.36,0.00}{\textbf{#1}}}
\newcommand{\SpecialStringTok}[1]{\textcolor[rgb]{0.31,0.60,0.02}{#1}}
\newcommand{\StringTok}[1]{\textcolor[rgb]{0.31,0.60,0.02}{#1}}
\newcommand{\VariableTok}[1]{\textcolor[rgb]{0.00,0.00,0.00}{#1}}
\newcommand{\VerbatimStringTok}[1]{\textcolor[rgb]{0.31,0.60,0.02}{#1}}
\newcommand{\WarningTok}[1]{\textcolor[rgb]{0.56,0.35,0.01}{\textbf{\textit{#1}}}}
\usepackage{longtable,booktabs,array}
\usepackage{calc} % for calculating minipage widths
% Correct order of tables after \paragraph or \subparagraph
\usepackage{etoolbox}
\makeatletter
\patchcmd\longtable{\par}{\if@noskipsec\mbox{}\fi\par}{}{}
\makeatother
% Allow footnotes in longtable head/foot
\IfFileExists{footnotehyper.sty}{\usepackage{footnotehyper}}{\usepackage{footnote}}
\makesavenoteenv{longtable}
\usepackage{graphicx}
\makeatletter
\def\maxwidth{\ifdim\Gin@nat@width>\linewidth\linewidth\else\Gin@nat@width\fi}
\def\maxheight{\ifdim\Gin@nat@height>\textheight\textheight\else\Gin@nat@height\fi}
\makeatother
% Scale images if necessary, so that they will not overflow the page
% margins by default, and it is still possible to overwrite the defaults
% using explicit options in \includegraphics[width, height, ...]{}
\setkeys{Gin}{width=\maxwidth,height=\maxheight,keepaspectratio}
% Set default figure placement to htbp
\makeatletter
\def\fps@figure{htbp}
\makeatother
\setlength{\emergencystretch}{3em} % prevent overfull lines
\providecommand{\tightlist}{%
  \setlength{\itemsep}{0pt}\setlength{\parskip}{0pt}}
\setcounter{secnumdepth}{5}
\usepackage{booktabs}
\ifLuaTeX
  \usepackage{selnolig}  % disable illegal ligatures
\fi
\usepackage[]{natbib}
\bibliographystyle{plainnat}
\IfFileExists{bookmark.sty}{\usepackage{bookmark}}{\usepackage{hyperref}}
\IfFileExists{xurl.sty}{\usepackage{xurl}}{} % add URL line breaks if available
\urlstyle{same}
\hypersetup{
  pdftitle={Intermediate Quantitative Methods},
  pdfauthor={Lucas Lemmann},
  hidelinks,
  pdfcreator={LaTeX via pandoc}}

\title{Intermediate Quantitative Methods}
\author{Lucas Lemmann}
\date{2023-10-23}

\begin{document}
\maketitle

{
\setcounter{tocdepth}{1}
\tableofcontents
}
\hypertarget{about}{%
\chapter*{About}\label{about}}
\addcontentsline{toc}{chapter}{About}

WHAT ABOUT IS THE BOOK?

\hypertarget{how-to-use-these-exercises}{%
\section{How to use these exercises?}\label{how-to-use-these-exercises}}

\begin{itemize}
\tightlist
\item
  Besides the 14 lectures, the course will be organized around 12 non-graded exercises:

  \begin{itemize}
  \tightlist
  \item
    5 labs
  \item
    7 do-it-yourself (DIYS)
  \end{itemize}
\item
  The labs' solutions will be discussed in detail between TAs and students in the corresponding sessions, while DIYS will not. In both cases, we will publish the solutions the week after the exercise is due.
\item
  We encourage you to prepare for the lab sessions in advance as well as to attend them to discuss any doubts they might have related to the labs material.
\item
  To prevent redundant communications (i.e., emails with the same information), share your questions regarding the exercises in the forum. Labs will emphasize the most voted questions.
\item
  While we encourage and foster a collaborative learning process, we expect you to work individually first.

  \begin{itemize}
  \tightlist
  \item
    I.e., try to address the task on your own first, identify what is limiting you, try to solve it on your own (not for too long), and, if you cannot find a solution, reach out your classmates. Once you find your solution, consider discussing the solution with your classmates.
  \end{itemize}
\end{itemize}

\hypertarget{schedule}{%
\section{Schedule}\label{schedule}}

\begin{longtable}[]{@{}ccc@{}}
\toprule\noalign{}
Week & Dates & Exercise type \\
\midrule\noalign{}
\endhead
\bottomrule\noalign{}
\endlastfoot
1 & 19-25/02 & DIYS 1 \\
2 & 26/02-03/03 & Lab 1 \\
3 & 04/03-10/03 & Lab 1 \\
4 & 11/03-17/03 & DIYS 2 \\
5 & 18/03-24/03 & Lab 2 \\
6 & 25/03-31/03 & DIYS 3 \\
\textbf{Spring Break} & 28/03-07/04 & None? \\
7 & 08/04-14/04 & Lab 3 \\
8 & 15/04-21/04 & DIYS 4 \\
9 & 22/04-28/04 & Lab 4 \\
10 & 29/04-05/05 & DIYS 5 \\
11 & 06/05-12/05 & Lab 5 \\
12 & 13/05-19/05 & Lab 5 \\
13 & 20/05-26/05 & DIYS 6 \\
14 & 27/05-02/06 & DIYS 7 \\
\end{longtable}

\hypertarget{week-1-diys-1}{%
\chapter{Week 1: DIYS 1}\label{week-1-diys-1}}

\hypertarget{aim}{%
\section{Aim:}\label{aim}}

To refresh your R skills by performing some basic analyses (i.e., descriptive, exploratory, and hypothesis testing ones).

\hypertarget{first-part-descriptive-analysis}{%
\section{First part: descriptive analysis}\label{first-part-descriptive-analysis}}

\begin{enumerate}
\def\labelenumi{\arabic{enumi}.}
\tightlist
\item
  Download the files \texttt{f.txt} and \texttt{m.txt}. They contain information on the number of steps in a day and the body mass index (BMI) for female and male individuals respectively. Open them and explore the first 5 observations for each file.
\end{enumerate}

{Adjust using the links from GitHub \href{}{}}

\begin{Shaded}
\begin{Highlighting}[]
\CommentTok{\# Your code goes here}
\end{Highlighting}
\end{Shaded}

{For the exercise before publishing the solution}

\begin{Shaded}
\begin{Highlighting}[]
\CommentTok{\# open data}
\NormalTok{female }\OtherTok{\textless{}{-}} \FunctionTok{read.table}\NormalTok{(}\StringTok{"\textasciitilde{}/Documents/0\_IPZ/2023\_2/Leemann{-}QuantMethods/QuantitativeMethods/QuantitativeMethods/Data/f.txt"}\NormalTok{, }\AttributeTok{header =} \ConstantTok{TRUE}\NormalTok{, }\AttributeTok{sep =} \StringTok{"}\SpecialCharTok{\textbackslash{}t}\StringTok{"}\NormalTok{)}

\CommentTok{\# explore data}
\FunctionTok{head}\NormalTok{(female, }\DecValTok{3}\NormalTok{)}
\end{Highlighting}
\end{Shaded}

\begin{verbatim}
##   ID steps  bmi
## 1  3 15000 17.0
## 2  4 14861 17.2
## 3  5 14861 17.2
\end{verbatim}

\begin{Shaded}
\begin{Highlighting}[]
\CommentTok{\# open data}
\NormalTok{male }\OtherTok{\textless{}{-}} \FunctionTok{read.table}\NormalTok{(}\StringTok{"\textasciitilde{}/Documents/0\_IPZ/2023\_2/Leemann{-}QuantMethods/QuantitativeMethods/QuantitativeMethods/Data/m.txt"}\NormalTok{, }\AttributeTok{header =} \ConstantTok{TRUE}\NormalTok{, }\AttributeTok{sep =} \StringTok{"}\SpecialCharTok{\textbackslash{}t}\StringTok{"}\NormalTok{)}

\CommentTok{\# explore data}
\FunctionTok{head}\NormalTok{(male, }\DecValTok{3}\NormalTok{)}
\end{Highlighting}
\end{Shaded}

\begin{verbatim}
##   ID steps  bmi
## 1  1 15000 16.9
## 2  2 15000 16.9
## 3  6 14861 16.8
\end{verbatim}

\begin{enumerate}
\def\labelenumi{\arabic{enumi}.}
\item
  Some key functions in dplyr can be categorized as dealing with columns (e.g., \texttt{select}, \texttt{mutate}), rows (e.g., \texttt{filter}, \texttt{distinct}, \texttt{arrange}), or groups (e.g., \texttt{group\_by}, \texttt{summarise}, and \texttt{count}). Let's use them:
\item
  Select only the columns `steps' and `bmi'. Do it only for the first three observations of the data on females.
\end{enumerate}

\begin{Shaded}
\begin{Highlighting}[]
\FunctionTok{library}\NormalTok{(dplyr)}
\end{Highlighting}
\end{Shaded}

\begin{verbatim}
## 
## Attaching package: 'dplyr'
\end{verbatim}

\begin{verbatim}
## The following objects are masked from 'package:stats':
## 
##     filter, lag
\end{verbatim}

\begin{verbatim}
## The following objects are masked from 'package:base':
## 
##     intersect, setdiff, setequal, union
\end{verbatim}

\begin{Shaded}
\begin{Highlighting}[]
\FunctionTok{head}\NormalTok{(female, }\DecValTok{3}\NormalTok{) }\SpecialCharTok{\%\textgreater{}\%}
  \FunctionTok{select}\NormalTok{(steps, bmi)}
\end{Highlighting}
\end{Shaded}

\begin{verbatim}
##   steps  bmi
## 1 15000 17.0
## 2 14861 17.2
## 3 14861 17.2
\end{verbatim}

\begin{enumerate}
\def\labelenumi{\arabic{enumi}.}
\setcounter{enumi}{1}
\tightlist
\item
  Select all columns except `ID'. Do not use \texttt{steps} nor \texttt{bmi}. Do it only for the first three observations of the data on females. Is the resulting table the same as the previous point? If not, check your answer.
\end{enumerate}

\begin{Shaded}
\begin{Highlighting}[]
\FunctionTok{library}\NormalTok{(dplyr)}
\FunctionTok{head}\NormalTok{(female, }\DecValTok{3}\NormalTok{) }\SpecialCharTok{\%\textgreater{}\%}
  \FunctionTok{select}\NormalTok{(}\SpecialCharTok{{-}}\NormalTok{ID)}
\end{Highlighting}
\end{Shaded}

\begin{verbatim}
##   steps  bmi
## 1 15000 17.0
## 2 14861 17.2
## 3 14861 17.2
\end{verbatim}

Note: to check the documentation of \texttt{select}, use \texttt{?select} on the console.

\begin{enumerate}
\def\labelenumi{\arabic{enumi}.}
\setcounter{enumi}{1}
\tightlist
\item
  Are there repeated ids within each data set?
\end{enumerate}

\begin{itemize}
\tightlist
\item
  Hint
\end{itemize}

\begin{Shaded}
\begin{Highlighting}[]
\CommentTok{\# get package}
\CommentTok{\# install.packages("dplyr")}
\FunctionTok{library}\NormalTok{(dplyr)}


\CommentTok{\# Check for repeated IDs in the female dataset}
\NormalTok{repeated\_ids\_female }\OtherTok{\textless{}{-}}\NormalTok{ female }\SpecialCharTok{\%\textgreater{}\%}
  \FunctionTok{group\_by}\NormalTok{(ID) }\SpecialCharTok{\%\textgreater{}\%}
  \FunctionTok{filter}\NormalTok{(}\FunctionTok{n}\NormalTok{() }\SpecialCharTok{\textgreater{}} \DecValTok{1}\NormalTok{)}

\FunctionTok{cat}\NormalTok{(}\StringTok{"Number of repeated IDs in the female dataset:"}\NormalTok{, }\FunctionTok{nrow}\NormalTok{(repeated\_ids\_female), }\StringTok{"}\SpecialCharTok{\textbackslash{}n}\StringTok{"}\NormalTok{)}
\end{Highlighting}
\end{Shaded}

\begin{verbatim}
## Number of repeated IDs in the female dataset: 0
\end{verbatim}

\begin{Shaded}
\begin{Highlighting}[]
\CommentTok{\# Check for repeated IDs in the male dataset}
\NormalTok{repeated\_ids\_male }\OtherTok{\textless{}{-}}\NormalTok{ male }\SpecialCharTok{\%\textgreater{}\%}
  \FunctionTok{group\_by}\NormalTok{(ID) }\SpecialCharTok{\%\textgreater{}\%}
  \FunctionTok{filter}\NormalTok{(}\FunctionTok{n}\NormalTok{() }\SpecialCharTok{\textgreater{}} \DecValTok{1}\NormalTok{)}

\FunctionTok{cat}\NormalTok{(}\StringTok{"Number of repeated IDs in the male dataset:"}\NormalTok{, }\FunctionTok{nrow}\NormalTok{(repeated\_ids\_male), }\StringTok{"}\SpecialCharTok{\textbackslash{}n}\StringTok{"}\NormalTok{)}
\end{Highlighting}
\end{Shaded}

\begin{verbatim}
## Number of repeated IDs in the male dataset: 0
\end{verbatim}

\begin{enumerate}
\def\labelenumi{\arabic{enumi}.}
\tightlist
\item
  Unify both data sets in one object.
\item
  Make sure you can distinguish the individual sex in the unified data set.
\item
  Consider using the packages \texttt{dplyr}, ``
\end{enumerate}

\hypertarget{solution}{%
\section{Solution}\label{solution}}

Will be made available.

\hypertarget{second-part}{%
\section{Second part:}\label{second-part}}

Please read the whole instruction before solving the exercise.

Each student will be randomly allocated to either doing the task 1 or 2 (a list containing those numbers will published). Both tasks are based on the same data sets. \texttt{f.txt} and \texttt{m.txt} contain information on the number of steps in a day and body mass index (BMI) for female and male individuals respectively.

Notes:

\begin{itemize}
\tightlist
\item
  The details of the data origin will be published with the solution.
\item
  Students allocated to each group are encouraged to do the task for the other group \emph{only} after finishing their own task.
\end{itemize}

\hypertarget{task-1}{%
\subsection{Task 1:}\label{task-1}}

\begin{itemize}
\tightlist
\item
  What do you conclude from the combined data set (i.e., the one formed using both the one for males and the one for females) regarding the relationship?
\item
  What questions did you ask yourself?

  \begin{itemize}
  \tightlist
  \item
    Why did you ask those questions? Is there an intuition behind them?

    \begin{itemize}
    \tightlist
    \item
      If so, what was your intuition?
    \item
      If not, how did you proceed?
    \end{itemize}
  \end{itemize}
\end{itemize}

\hypertarget{task-2}{%
\subsection{Task 2:}\label{task-2}}

\begin{itemize}
\tightlist
\item
  Is the average number of steps for males and females statistically different?
\item
  How do BMI and daily steps statistically relate to each other?

  \begin{itemize}
  \tightlist
  \item
    Does that relationship depend on whether individuals are of one sex or another? If so, how?

    \begin{itemize}
    \tightlist
    \item
      Is there an statistically significant negative correlation between the number of steps and the BMI for females?
    \item
      Is there an statistically significant positive correlation between the number of steps and the BMI for males?
    \end{itemize}
  \end{itemize}
\item
  1st weeks, dplier: to check\textgreater{} to statistical analysis

  \begin{itemize}
  \tightlist
  \item
    Doing basic code to make analysis (which is fine enough), but in dplier you could do it like this.
  \item
    Make descriptive statistics using an interesting
  \end{itemize}
\end{itemize}

looking for something unknown in the dark, grope, feel blindly and make conjectures on what things are and how they are related.
- Two groups: random selection: description similar? The smaller the group, the likelier that a random selection is not balanced? What about attrition?

Looking!=seeing:
Different beliefs (non- and knowledge ones), different preferences, different attention focus -\textgreater{} different attention investment and emphasis
Value of diverse academic community while keeping a minimal set of shared assessment rules: \href{https://plato.stanford.edu/entries/scientific-objectivity/}{objectivity} as continuum of increasing inter-subjective agreement

\hypertarget{week-2}{%
\chapter{Week 2}\label{week-2}}

\hypertarget{exercise}{%
\section{Exercise}\label{exercise}}

\begin{itemize}
\tightlist
\item
  2nd: simulated dataset and increase the variance: how does that affects the standard error
\end{itemize}

\hypertarget{solution-1}{%
\section{Solution}\label{solution-1}}

\begin{itemize}
\item
  Data taken from \href{https://communities.sas.com/t5/Graphics-Programming/Fun-With-SAS-ODS-Graphics-Don-t-Miss-the-Gorilla-in-the-Data/td-p/697286}{here}.
\item
  Original selective attention, \href{https://www.youtube.com/watch?v=vJG698U2Mvo}{here}.
\item
  Suicide awareness campaign, \href{https://www.youtube.com/watch?v=Lw-YPKR0grk}{here}.
\end{itemize}

\hypertarget{week-3}{%
\chapter{Week 3}\label{week-3}}

\hypertarget{exercise-1}{%
\section{Exercise}\label{exercise-1}}

\hypertarget{solution-2}{%
\section{Solution}\label{solution-2}}

\hypertarget{week-4}{%
\chapter{Week 4}\label{week-4}}

\hypertarget{exercise-2}{%
\section{Exercise}\label{exercise-2}}

\hypertarget{solution-3}{%
\section{Solution}\label{solution-3}}

\hypertarget{week-5}{%
\chapter{Week 5}\label{week-5}}

\hypertarget{exercise-3}{%
\section{Exercise}\label{exercise-3}}

\hypertarget{solution-4}{%
\section{Solution}\label{solution-4}}

\hypertarget{week-6}{%
\chapter{Week 6}\label{week-6}}

\hypertarget{exercise-4}{%
\section{Exercise}\label{exercise-4}}

\hypertarget{solution-5}{%
\section{Solution}\label{solution-5}}

\hypertarget{week-7}{%
\chapter{Week 7}\label{week-7}}

\hypertarget{exercise-5}{%
\section{Exercise}\label{exercise-5}}

\hypertarget{solution-6}{%
\section{Solution}\label{solution-6}}

\hypertarget{week-8}{%
\chapter{Week 8}\label{week-8}}

\hypertarget{exercise-6}{%
\section{Exercise}\label{exercise-6}}

\hypertarget{solution-7}{%
\section{Solution}\label{solution-7}}

\hypertarget{week-9}{%
\chapter{Week 9}\label{week-9}}

\hypertarget{exercise-7}{%
\section{Exercise}\label{exercise-7}}

\hypertarget{solution-8}{%
\section{Solution}\label{solution-8}}

\hypertarget{week-10}{%
\chapter{Week 10}\label{week-10}}

\hypertarget{exercise-8}{%
\section{Exercise}\label{exercise-8}}

\hypertarget{solution-9}{%
\section{Solution}\label{solution-9}}

\hypertarget{week-11}{%
\chapter{Week 11}\label{week-11}}

\hypertarget{exercise-9}{%
\section{Exercise}\label{exercise-9}}

\hypertarget{solution-10}{%
\section{Solution}\label{solution-10}}

\hypertarget{week-12}{%
\chapter{Week 12}\label{week-12}}

\hypertarget{exercise-10}{%
\section{Exercise}\label{exercise-10}}

\hypertarget{solution-11}{%
\section{Solution}\label{solution-11}}

\hypertarget{week-13}{%
\chapter{Week 13}\label{week-13}}

\hypertarget{exercise-11}{%
\section{Exercise}\label{exercise-11}}

\hypertarget{solution-12}{%
\section{Solution}\label{solution-12}}

\hypertarget{week-14}{%
\chapter{Week 14}\label{week-14}}

\hypertarget{exercise-12}{%
\section{Exercise}\label{exercise-12}}

\hypertarget{solution-13}{%
\section{Solution}\label{solution-13}}

  \bibliography{book.bib,packages.bib}

\end{document}
